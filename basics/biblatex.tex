% biblatex settings
% ... extracted from `pkuthss`
% ... biber usage:
%%%% !TeX TXS-program:bibliography = biber -l zh__pinyin --output-safechars %
%%%% % 注意末尾的 %, 适用于 TeXstudio

%% ... include bib:
%\addbibresource{*.bib}
%% ... print bib:
%\printbibliography[heading = bibintoc, title = {参考文献}]
%% bibintoc 选项使“参考文献”出现在目录中
%% 如果同时要使参考文献列表参与章节编号,
%% ... 可将“bibintoc”改为“bibnumbered”

\usepackage[
%	utf8
%	,style=caspervector
	,style=trad-unsrt
	,citestyle=numeric-comp
	,backend=biber % caspervector 必须使用 biber 后端
	,sorting=none % 英、中文献排序,对比 ecnyt, cenyt
%	,block=ragged
]{biblatex}
% 按学校要求设定参考文献列表中的条目之内及之间的距离
\setlength{\bibitemsep}{3bp}
% linespread 值的计算过程可以参考 pkuthss.cls
\renewcommand*{\bibfont}{\zihao{5}\linespread{1.27}\selectfont}

\let\simplecite\cite
\renewcommand{\cite}[1]{\parencite{#1}}

% DOI as hyperlink & titles
\ExecuteBibliographyOptions{doi=false}
\newbibmacro{string+doi}[1]{%
	\iffieldundef{doi}{#1}%
	{\href{https://dx.doi.org/\thefield{doi}}{#1}}%
}
\DeclareFieldFormat*{title}%
	{\usebibmacro{string+doi}{\mkbibemph{#1}}\isdot}
\DeclareFieldFormat*{journaltitle}%
	{\mkbibemph{#1}\adddot\nopunct}
\DeclareFieldFormat{titlecase}{#1} % no lower case (SentenceCase)

\DeclareFieldFormat[article]{volume}%
	{\mkbibbold{#1}\isdot}
\DeclareFieldFormat{date}%
	{\mkbibbold{#1}\isdot}
\DeclareFieldFormat{isbn}%
	{\mkbibacro{ISBN}\addcolon\addnbspace\smaller[0.5]#1\isdot}
\DeclareFieldFormat{url}%
	{\mkbibacro{URL}\addcolon\addnbspace\smaller[0.5]\url{#1}\isdot}
